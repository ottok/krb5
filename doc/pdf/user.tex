% Generated by Sphinx.
\def\sphinxdocclass{report}
\documentclass[letterpaper,10pt,english]{sphinxmanual}
\usepackage[utf8]{inputenc}
\DeclareUnicodeCharacter{00A0}{\nobreakspace}
\usepackage{cmap}
\usepackage[T1]{fontenc}
\usepackage{babel}
\usepackage{times}
\usepackage[Bjarne]{fncychap}
\usepackage{longtable}
\usepackage{sphinx}
\usepackage{multirow}


\title{Kerberos User Guide}
\date{ }
\release{1.15-beta1}
\author{MIT}
\newcommand{\sphinxlogo}{}
\renewcommand{\releasename}{Release}
\makeindex

\makeatletter
\def\PYG@reset{\let\PYG@it=\relax \let\PYG@bf=\relax%
    \let\PYG@ul=\relax \let\PYG@tc=\relax%
    \let\PYG@bc=\relax \let\PYG@ff=\relax}
\def\PYG@tok#1{\csname PYG@tok@#1\endcsname}
\def\PYG@toks#1+{\ifx\relax#1\empty\else%
    \PYG@tok{#1}\expandafter\PYG@toks\fi}
\def\PYG@do#1{\PYG@bc{\PYG@tc{\PYG@ul{%
    \PYG@it{\PYG@bf{\PYG@ff{#1}}}}}}}
\def\PYG#1#2{\PYG@reset\PYG@toks#1+\relax+\PYG@do{#2}}

\expandafter\def\csname PYG@tok@gd\endcsname{\def\PYG@tc##1{\textcolor[rgb]{0.63,0.00,0.00}{##1}}}
\expandafter\def\csname PYG@tok@gu\endcsname{\let\PYG@bf=\textbf\def\PYG@tc##1{\textcolor[rgb]{0.50,0.00,0.50}{##1}}}
\expandafter\def\csname PYG@tok@gt\endcsname{\def\PYG@tc##1{\textcolor[rgb]{0.00,0.27,0.87}{##1}}}
\expandafter\def\csname PYG@tok@gs\endcsname{\let\PYG@bf=\textbf}
\expandafter\def\csname PYG@tok@gr\endcsname{\def\PYG@tc##1{\textcolor[rgb]{1.00,0.00,0.00}{##1}}}
\expandafter\def\csname PYG@tok@cm\endcsname{\let\PYG@it=\textit\def\PYG@tc##1{\textcolor[rgb]{0.25,0.50,0.56}{##1}}}
\expandafter\def\csname PYG@tok@vg\endcsname{\def\PYG@tc##1{\textcolor[rgb]{0.73,0.38,0.84}{##1}}}
\expandafter\def\csname PYG@tok@m\endcsname{\def\PYG@tc##1{\textcolor[rgb]{0.13,0.50,0.31}{##1}}}
\expandafter\def\csname PYG@tok@mh\endcsname{\def\PYG@tc##1{\textcolor[rgb]{0.13,0.50,0.31}{##1}}}
\expandafter\def\csname PYG@tok@cs\endcsname{\def\PYG@tc##1{\textcolor[rgb]{0.25,0.50,0.56}{##1}}\def\PYG@bc##1{\setlength{\fboxsep}{0pt}\colorbox[rgb]{1.00,0.94,0.94}{\strut ##1}}}
\expandafter\def\csname PYG@tok@ge\endcsname{\let\PYG@it=\textit}
\expandafter\def\csname PYG@tok@vc\endcsname{\def\PYG@tc##1{\textcolor[rgb]{0.73,0.38,0.84}{##1}}}
\expandafter\def\csname PYG@tok@il\endcsname{\def\PYG@tc##1{\textcolor[rgb]{0.13,0.50,0.31}{##1}}}
\expandafter\def\csname PYG@tok@go\endcsname{\def\PYG@tc##1{\textcolor[rgb]{0.20,0.20,0.20}{##1}}}
\expandafter\def\csname PYG@tok@cp\endcsname{\def\PYG@tc##1{\textcolor[rgb]{0.00,0.44,0.13}{##1}}}
\expandafter\def\csname PYG@tok@gi\endcsname{\def\PYG@tc##1{\textcolor[rgb]{0.00,0.63,0.00}{##1}}}
\expandafter\def\csname PYG@tok@gh\endcsname{\let\PYG@bf=\textbf\def\PYG@tc##1{\textcolor[rgb]{0.00,0.00,0.50}{##1}}}
\expandafter\def\csname PYG@tok@ni\endcsname{\let\PYG@bf=\textbf\def\PYG@tc##1{\textcolor[rgb]{0.84,0.33,0.22}{##1}}}
\expandafter\def\csname PYG@tok@nl\endcsname{\let\PYG@bf=\textbf\def\PYG@tc##1{\textcolor[rgb]{0.00,0.13,0.44}{##1}}}
\expandafter\def\csname PYG@tok@nn\endcsname{\let\PYG@bf=\textbf\def\PYG@tc##1{\textcolor[rgb]{0.05,0.52,0.71}{##1}}}
\expandafter\def\csname PYG@tok@no\endcsname{\def\PYG@tc##1{\textcolor[rgb]{0.38,0.68,0.84}{##1}}}
\expandafter\def\csname PYG@tok@na\endcsname{\def\PYG@tc##1{\textcolor[rgb]{0.25,0.44,0.63}{##1}}}
\expandafter\def\csname PYG@tok@nb\endcsname{\def\PYG@tc##1{\textcolor[rgb]{0.00,0.44,0.13}{##1}}}
\expandafter\def\csname PYG@tok@nc\endcsname{\let\PYG@bf=\textbf\def\PYG@tc##1{\textcolor[rgb]{0.05,0.52,0.71}{##1}}}
\expandafter\def\csname PYG@tok@nd\endcsname{\let\PYG@bf=\textbf\def\PYG@tc##1{\textcolor[rgb]{0.33,0.33,0.33}{##1}}}
\expandafter\def\csname PYG@tok@ne\endcsname{\def\PYG@tc##1{\textcolor[rgb]{0.00,0.44,0.13}{##1}}}
\expandafter\def\csname PYG@tok@nf\endcsname{\def\PYG@tc##1{\textcolor[rgb]{0.02,0.16,0.49}{##1}}}
\expandafter\def\csname PYG@tok@si\endcsname{\let\PYG@it=\textit\def\PYG@tc##1{\textcolor[rgb]{0.44,0.63,0.82}{##1}}}
\expandafter\def\csname PYG@tok@s2\endcsname{\def\PYG@tc##1{\textcolor[rgb]{0.25,0.44,0.63}{##1}}}
\expandafter\def\csname PYG@tok@vi\endcsname{\def\PYG@tc##1{\textcolor[rgb]{0.73,0.38,0.84}{##1}}}
\expandafter\def\csname PYG@tok@nt\endcsname{\let\PYG@bf=\textbf\def\PYG@tc##1{\textcolor[rgb]{0.02,0.16,0.45}{##1}}}
\expandafter\def\csname PYG@tok@nv\endcsname{\def\PYG@tc##1{\textcolor[rgb]{0.73,0.38,0.84}{##1}}}
\expandafter\def\csname PYG@tok@s1\endcsname{\def\PYG@tc##1{\textcolor[rgb]{0.25,0.44,0.63}{##1}}}
\expandafter\def\csname PYG@tok@gp\endcsname{\let\PYG@bf=\textbf\def\PYG@tc##1{\textcolor[rgb]{0.78,0.36,0.04}{##1}}}
\expandafter\def\csname PYG@tok@sh\endcsname{\def\PYG@tc##1{\textcolor[rgb]{0.25,0.44,0.63}{##1}}}
\expandafter\def\csname PYG@tok@ow\endcsname{\let\PYG@bf=\textbf\def\PYG@tc##1{\textcolor[rgb]{0.00,0.44,0.13}{##1}}}
\expandafter\def\csname PYG@tok@sx\endcsname{\def\PYG@tc##1{\textcolor[rgb]{0.78,0.36,0.04}{##1}}}
\expandafter\def\csname PYG@tok@bp\endcsname{\def\PYG@tc##1{\textcolor[rgb]{0.00,0.44,0.13}{##1}}}
\expandafter\def\csname PYG@tok@c1\endcsname{\let\PYG@it=\textit\def\PYG@tc##1{\textcolor[rgb]{0.25,0.50,0.56}{##1}}}
\expandafter\def\csname PYG@tok@kc\endcsname{\let\PYG@bf=\textbf\def\PYG@tc##1{\textcolor[rgb]{0.00,0.44,0.13}{##1}}}
\expandafter\def\csname PYG@tok@c\endcsname{\let\PYG@it=\textit\def\PYG@tc##1{\textcolor[rgb]{0.25,0.50,0.56}{##1}}}
\expandafter\def\csname PYG@tok@mf\endcsname{\def\PYG@tc##1{\textcolor[rgb]{0.13,0.50,0.31}{##1}}}
\expandafter\def\csname PYG@tok@err\endcsname{\def\PYG@bc##1{\setlength{\fboxsep}{0pt}\fcolorbox[rgb]{1.00,0.00,0.00}{1,1,1}{\strut ##1}}}
\expandafter\def\csname PYG@tok@kd\endcsname{\let\PYG@bf=\textbf\def\PYG@tc##1{\textcolor[rgb]{0.00,0.44,0.13}{##1}}}
\expandafter\def\csname PYG@tok@ss\endcsname{\def\PYG@tc##1{\textcolor[rgb]{0.32,0.47,0.09}{##1}}}
\expandafter\def\csname PYG@tok@sr\endcsname{\def\PYG@tc##1{\textcolor[rgb]{0.14,0.33,0.53}{##1}}}
\expandafter\def\csname PYG@tok@mo\endcsname{\def\PYG@tc##1{\textcolor[rgb]{0.13,0.50,0.31}{##1}}}
\expandafter\def\csname PYG@tok@mi\endcsname{\def\PYG@tc##1{\textcolor[rgb]{0.13,0.50,0.31}{##1}}}
\expandafter\def\csname PYG@tok@kn\endcsname{\let\PYG@bf=\textbf\def\PYG@tc##1{\textcolor[rgb]{0.00,0.44,0.13}{##1}}}
\expandafter\def\csname PYG@tok@o\endcsname{\def\PYG@tc##1{\textcolor[rgb]{0.40,0.40,0.40}{##1}}}
\expandafter\def\csname PYG@tok@kr\endcsname{\let\PYG@bf=\textbf\def\PYG@tc##1{\textcolor[rgb]{0.00,0.44,0.13}{##1}}}
\expandafter\def\csname PYG@tok@s\endcsname{\def\PYG@tc##1{\textcolor[rgb]{0.25,0.44,0.63}{##1}}}
\expandafter\def\csname PYG@tok@kp\endcsname{\def\PYG@tc##1{\textcolor[rgb]{0.00,0.44,0.13}{##1}}}
\expandafter\def\csname PYG@tok@w\endcsname{\def\PYG@tc##1{\textcolor[rgb]{0.73,0.73,0.73}{##1}}}
\expandafter\def\csname PYG@tok@kt\endcsname{\def\PYG@tc##1{\textcolor[rgb]{0.56,0.13,0.00}{##1}}}
\expandafter\def\csname PYG@tok@sc\endcsname{\def\PYG@tc##1{\textcolor[rgb]{0.25,0.44,0.63}{##1}}}
\expandafter\def\csname PYG@tok@sb\endcsname{\def\PYG@tc##1{\textcolor[rgb]{0.25,0.44,0.63}{##1}}}
\expandafter\def\csname PYG@tok@k\endcsname{\let\PYG@bf=\textbf\def\PYG@tc##1{\textcolor[rgb]{0.00,0.44,0.13}{##1}}}
\expandafter\def\csname PYG@tok@se\endcsname{\let\PYG@bf=\textbf\def\PYG@tc##1{\textcolor[rgb]{0.25,0.44,0.63}{##1}}}
\expandafter\def\csname PYG@tok@sd\endcsname{\let\PYG@it=\textit\def\PYG@tc##1{\textcolor[rgb]{0.25,0.44,0.63}{##1}}}

\def\PYGZbs{\char`\\}
\def\PYGZus{\char`\_}
\def\PYGZob{\char`\{}
\def\PYGZcb{\char`\}}
\def\PYGZca{\char`\^}
\def\PYGZam{\char`\&}
\def\PYGZlt{\char`\<}
\def\PYGZgt{\char`\>}
\def\PYGZsh{\char`\#}
\def\PYGZpc{\char`\%}
\def\PYGZdl{\char`\$}
\def\PYGZhy{\char`\-}
\def\PYGZsq{\char`\'}
\def\PYGZdq{\char`\"}
\def\PYGZti{\char`\~}
% for compatibility with earlier versions
\def\PYGZat{@}
\def\PYGZlb{[}
\def\PYGZrb{]}
\makeatother

\begin{document}

\maketitle
\tableofcontents
\phantomsection\label{user/index::doc}



\chapter{Password management}
\label{user/pwd_mgmt:for-users}\label{user/pwd_mgmt::doc}\label{user/pwd_mgmt:password-management}
Your password is the only way Kerberos has of verifying your identity.
If someone finds out your password, that person can masquerade as
you---send email that comes from you, read, edit, or delete your files,
or log into other hosts as you---and no one will be able to tell the
difference.  For this reason, it is important that you choose a good
password, and keep it secret.  If you need to give access to your
account to someone else, you can do so through Kerberos (see
{\hyperref[user/pwd_mgmt:grant-access]{\emph{Granting access to your account}}}).  You should never tell your password to anyone,
including your system administrator, for any reason.  You should
change your password frequently, particularly any time you think
someone may have found out what it is.


\section{Changing your password}
\label{user/pwd_mgmt:changing-your-password}
To change your Kerberos password, use the {\hyperref[user/user_commands/kpasswd:kpasswd-1]{\emph{kpasswd}}} command.
It will ask you for your old password (to prevent someone else from
walking up to your computer when you're not there and changing your
password), and then prompt you for the new one twice.  (The reason you
have to type it twice is to make sure you have typed it correctly.)
For example, user \code{david} would do the following:

\begin{Verbatim}[commandchars=\\\{\}]
shell\PYGZpc{} kpasswd
Password for david:    \PYGZlt{}\PYGZhy{} Type your old password.
Enter new password:    \PYGZlt{}\PYGZhy{} Type your new password.
Enter it again:  \PYGZlt{}\PYGZhy{} Type the new password again.
Password changed.
shell\PYGZpc{}
\end{Verbatim}

If \code{david} typed the incorrect old password, he would get the
following message:

\begin{Verbatim}[commandchars=\\\{\}]
shell\PYGZpc{} kpasswd
Password for david:  \PYGZlt{}\PYGZhy{} Type the incorrect old password.
kpasswd: Password incorrect while getting initial ticket
shell\PYGZpc{}
\end{Verbatim}

If you make a mistake and don't type the new password the same way
twice, kpasswd will ask you to try again:

\begin{Verbatim}[commandchars=\\\{\}]
shell\PYGZpc{} kpasswd
Password for david:  \PYGZlt{}\PYGZhy{} Type the old password.
Enter new password:  \PYGZlt{}\PYGZhy{} Type the new password.
Enter it again: \PYGZlt{}\PYGZhy{} Type a different new password.
kpasswd: Password mismatch while reading password
shell\PYGZpc{}
\end{Verbatim}

Once you change your password, it takes some time for the change to
propagate through the system.  Depending on how your system is set up,
this might be anywhere from a few minutes to an hour or more.  If you
need to get new Kerberos tickets shortly after changing your password,
try the new password.  If the new password doesn't work, try again
using the old one.


\section{Granting access to your account}
\label{user/pwd_mgmt:grant-access}\label{user/pwd_mgmt:granting-access-to-your-account}
If you need to give someone access to log into your account, you can
do so through Kerberos, without telling the person your password.
Simply create a file called {\hyperref[user/user_config/k5login:k5login-5]{\emph{.k5login}}} in your home directory.
This file should contain the Kerberos principal of each person to whom
you wish to give access.  Each principal must be on a separate line.
Here is a sample .k5login file:

\begin{Verbatim}[commandchars=\\\{\}]
jennifer@ATHENA.MIT.EDU
david@EXAMPLE.COM
\end{Verbatim}

This file would allow the users \code{jennifer} and \code{david} to use your
user ID, provided that they had Kerberos tickets in their respective
realms.  If you will be logging into other hosts across a network, you
will want to include your own Kerberos principal in your .k5login file
on each of these hosts.

Using a .k5login file is much safer than giving out your password,
because:
\begin{itemize}
\item {} 
You can take access away any time simply by removing the principal
from your .k5login file.

\item {} 
Although the user has full access to your account on one particular
host (or set of hosts if your .k5login file is shared, e.g., over
NFS), that user does not inherit your network privileges.

\item {} 
Kerberos keeps a log of who obtains tickets, so a system
administrator could find out, if necessary, who was capable of using
your user ID at a particular time.

\end{itemize}

One common application is to have a .k5login file in root's home
directory, giving root access to that machine to the Kerberos
principals listed.  This allows system administrators to allow users
to become root locally, or to log in remotely as root, without their
having to give out the root password, and without anyone having to
type the root password over the network.


\section{Password quality verification}
\label{user/pwd_mgmt:password-quality-verification}
TODO


\chapter{Ticket management}
\label{user/tkt_mgmt:ticket-management}\label{user/tkt_mgmt::doc}
On many systems, Kerberos is built into the login program, and you get
tickets automatically when you log in.  Other programs, such as ssh,
can forward copies of your tickets to a remote host.  Most of these
programs also automatically destroy your tickets when they exit.
However, MIT recommends that you explicitly destroy your Kerberos
tickets when you are through with them, just to be sure.  One way to
help ensure that this happens is to add the {\hyperref[user/user_commands/kdestroy:kdestroy-1]{\emph{kdestroy}}} command
to your .logout file.  Additionally, if you are going to be away from
your machine and are concerned about an intruder using your
permissions, it is safest to either destroy all copies of your
tickets, or use a screensaver that locks the screen.


\section{Kerberos ticket properties}
\label{user/tkt_mgmt:kerberos-ticket-properties}
There are various properties that Kerberos tickets can have:

If a ticket is \textbf{forwardable}, then the KDC can issue a new ticket
(with a different network address, if necessary) based on the
forwardable ticket.  This allows for authentication forwarding without
requiring a password to be typed in again.  For example, if a user
with a forwardable TGT logs into a remote system, the KDC could issue
a new TGT for that user with the network address of the remote system,
allowing authentication on that host to work as though the user were
logged in locally.

When the KDC creates a new ticket based on a forwardable ticket, it
sets the \textbf{forwarded} flag on that new ticket.  Any tickets that are
created based on a ticket with the forwarded flag set will also have
their forwarded flags set.

A \textbf{proxiable} ticket is similar to a forwardable ticket in that it
allows a service to take on the identity of the client.  Unlike a
forwardable ticket, however, a proxiable ticket is only issued for
specific services.  In other words, a ticket-granting ticket cannot be
issued based on a ticket that is proxiable but not forwardable.

A \textbf{proxy} ticket is one that was issued based on a proxiable ticket.

A \textbf{postdated} ticket is issued with the invalid flag set.  After the
starting time listed on the ticket, it can be presented to the KDC to
obtain valid tickets.

Ticket-granting tickets with the \textbf{postdateable} flag set can be used
to obtain postdated service tickets.

\textbf{Renewable} tickets can be used to obtain new session keys without
the user entering their password again.  A renewable ticket has two
expiration times.  The first is the time at which this particular
ticket expires.  The second is the latest possible expiration time for
any ticket issued based on this renewable ticket.

A ticket with the \textbf{initial flag} set was issued based on the
authentication protocol, and not on a ticket-granting ticket.
Application servers that wish to ensure that the user's key has been
recently presented for verification could specify that this flag must
be set to accept the ticket.

An \textbf{invalid} ticket must be rejected by application servers.
Postdated tickets are usually issued with this flag set, and must be
validated by the KDC before they can be used.

A \textbf{preauthenticated} ticket is one that was only issued after the
client requesting the ticket had authenticated itself to the KDC.

The \textbf{hardware authentication} flag is set on a ticket which required
the use of hardware for authentication.  The hardware is expected to
be possessed only by the client which requested the tickets.

If a ticket has the \textbf{transit policy} checked flag set, then the KDC
that issued this ticket implements the transited-realm check policy
and checked the transited-realms list on the ticket.  The
transited-realms list contains a list of all intermediate realms
between the realm of the KDC that issued the first ticket and that of
the one that issued the current ticket.  If this flag is not set, then
the application server must check the transited realms itself or else
reject the ticket.

The \textbf{okay as delegate} flag indicates that the server specified in
the ticket is suitable as a delegate as determined by the policy of
that realm.  Some client applications may use this flag to decide
whether to forward tickets to a remote host, although many
applications do not honor it.

An \textbf{anonymous} ticket is one in which the named principal is a
generic principal for that realm; it does not actually specify the
individual that will be using the ticket.  This ticket is meant only
to securely distribute a session key.


\section{Obtaining tickets with kinit}
\label{user/tkt_mgmt:obtaining-tickets-with-kinit}\label{user/tkt_mgmt:obtain-tkt}
If your site has integrated Kerberos V5 with the login system, you
will get Kerberos tickets automatically when you log in.  Otherwise,
you may need to explicitly obtain your Kerberos tickets, using the
{\hyperref[user/user_commands/kinit:kinit-1]{\emph{kinit}}} program.  Similarly, if your Kerberos tickets expire,
use the kinit program to obtain new ones.

To use the kinit program, simply type \code{kinit} and then type your
password at the prompt. For example, Jennifer (whose username is
\code{jennifer}) works for Bleep, Inc. (a fictitious company with the
domain name mit.edu and the Kerberos realm ATHENA.MIT.EDU).  She would
type:

\begin{Verbatim}[commandchars=\\\{\}]
shell\PYGZpc{} kinit
Password for jennifer@ATHENA.MIT.EDU: \PYGZlt{}\PYGZhy{}\PYGZhy{} [Type jennifer\PYGZsq{}s password here.]
shell\PYGZpc{}
\end{Verbatim}

If you type your password incorrectly, kinit will give you the
following error message:

\begin{Verbatim}[commandchars=\\\{\}]
shell\PYGZpc{} kinit
Password for jennifer@ATHENA.MIT.EDU: \PYGZlt{}\PYGZhy{}\PYGZhy{} [Type the wrong password here.]
kinit: Password incorrect
shell\PYGZpc{}
\end{Verbatim}

and you won't get Kerberos tickets.

By default, kinit assumes you want tickets for your own username in
your default realm.  Suppose Jennifer's friend David is visiting, and
he wants to borrow a window to check his mail.  David needs to get
tickets for himself in his own realm, EXAMPLE.COM.  He would type:

\begin{Verbatim}[commandchars=\\\{\}]
shell\PYGZpc{} kinit david@EXAMPLE.COM
Password for david@EXAMPLE.COM: \PYGZlt{}\PYGZhy{}\PYGZhy{} [Type david\PYGZsq{}s password here.]
shell\PYGZpc{}
\end{Verbatim}

David would then have tickets which he could use to log onto his own
machine.  Note that he typed his password locally on Jennifer's
machine, but it never went over the network.  Kerberos on the local
host performed the authentication to the KDC in the other realm.

If you want to be able to forward your tickets to another host, you
need to request forwardable tickets.  You do this by specifying the
\textbf{-f} option:

\begin{Verbatim}[commandchars=\\\{\}]
shell\PYGZpc{} kinit \PYGZhy{}f
Password for jennifer@ATHENA.MIT.EDU: \PYGZlt{}\PYGZhy{}\PYGZhy{} [Type your password here.]
shell\PYGZpc{}
\end{Verbatim}

Note that kinit does not tell you that it obtained forwardable
tickets; you can verify this using the {\hyperref[user/user_commands/klist:klist-1]{\emph{klist}}} command (see
{\hyperref[user/tkt_mgmt:view-tkt]{\emph{Viewing tickets with klist}}}).

Normally, your tickets are good for your system's default ticket
lifetime, which is ten hours on many systems.  You can specify a
different ticket lifetime with the \textbf{-l} option.  Add the letter
\textbf{s} to the value for seconds, \textbf{m} for minutes, \textbf{h} for hours, or
\textbf{d} for days.  For example, to obtain forwardable tickets for
\code{david@EXAMPLE.COM} that would be good for three hours, you would
type:

\begin{Verbatim}[commandchars=\\\{\}]
shell\PYGZpc{} kinit \PYGZhy{}f \PYGZhy{}l 3h david@EXAMPLE.COM
Password for david@EXAMPLE.COM: \PYGZlt{}\PYGZhy{}\PYGZhy{} [Type david\PYGZsq{}s password here.]
shell\PYGZpc{}
\end{Verbatim}

\begin{notice}{note}{Note:}
You cannot mix units; specifying a lifetime of 3h30m would
result in an error.  Note also that most systems specify a
maximum ticket lifetime.  If you request a longer ticket
lifetime, it will be automatically truncated to the maximum
lifetime.
\end{notice}


\section{Viewing tickets with klist}
\label{user/tkt_mgmt:viewing-tickets-with-klist}\label{user/tkt_mgmt:view-tkt}
The {\hyperref[user/user_commands/klist:klist-1]{\emph{klist}}} command shows your tickets.  When you first obtain
tickets, you will have only the ticket-granting ticket.  The listing
would look like this:

\begin{Verbatim}[commandchars=\\\{\}]
shell\PYGZpc{} klist
Ticket cache: /tmp/krb5cc\PYGZus{}ttypa
Default principal: jennifer@ATHENA.MIT.EDU

Valid starting     Expires            Service principal
06/07/04 19:49:21  06/08/04 05:49:19  krbtgt/ATHENA.MIT.EDU@ATHENA.MIT.EDU
shell\PYGZpc{}
\end{Verbatim}

The ticket cache is the location of your ticket file. In the above
example, this file is named \code{/tmp/krb5cc\_ttypa}. The default
principal is your Kerberos principal.

The ``valid starting'' and ``expires'' fields describe the period of time
during which the ticket is valid.  The ``service principal'' describes
each ticket.  The ticket-granting ticket has a first component
\code{krbtgt}, and a second component which is the realm name.

Now, if \code{jennifer} connected to the machine \code{daffodil.mit.edu},
and then typed ``klist'' again, she would have gotten the following
result:

\begin{Verbatim}[commandchars=\\\{\}]
shell\PYGZpc{} klist
Ticket cache: /tmp/krb5cc\PYGZus{}ttypa
Default principal: jennifer@ATHENA.MIT.EDU

Valid starting     Expires            Service principal
06/07/04 19:49:21  06/08/04 05:49:19  krbtgt/ATHENA.MIT.EDU@ATHENA.MIT.EDU
06/07/04 20:22:30  06/08/04 05:49:19  host/daffodil.mit.edu@ATHENA.MIT.EDU
shell\PYGZpc{}
\end{Verbatim}

Here's what happened: when \code{jennifer} used ssh to connect to the
host \code{daffodil.mit.edu}, the ssh program presented her
ticket-granting ticket to the KDC and requested a host ticket for the
host \code{daffodil.mit.edu}.  The KDC sent the host ticket, which ssh
then presented to the host \code{daffodil.mit.edu}, and she was allowed
to log in without typing her password.

Suppose your Kerberos tickets allow you to log into a host in another
domain, such as \code{trillium.example.com}, which is also in another
Kerberos realm, \code{EXAMPLE.COM}.  If you ssh to this host, you will
receive a ticket-granting ticket for the realm \code{EXAMPLE.COM}, plus
the new host ticket for \code{trillium.example.com}.  klist will now
show:

\begin{Verbatim}[commandchars=\\\{\}]
shell\PYGZpc{} klist
Ticket cache: /tmp/krb5cc\PYGZus{}ttypa
Default principal: jennifer@ATHENA.MIT.EDU

Valid starting     Expires            Service principal
06/07/04 19:49:21  06/08/04 05:49:19  krbtgt/ATHENA.MIT.EDU@ATHENA.MIT.EDU
06/07/04 20:22:30  06/08/04 05:49:19  host/daffodil.mit.edu@ATHENA.MIT.EDU
06/07/04 20:24:18  06/08/04 05:49:19  krbtgt/EXAMPLE.COM@ATHENA.MIT.EDU
06/07/04 20:24:18  06/08/04 05:49:19  host/trillium.example.com@EXAMPLE.COM
shell\PYGZpc{}
\end{Verbatim}

Depending on your host's and realm's configuration, you may also see a
ticket with the service principal \code{host/trillium.example.com@}.  If
so, this means that your host did not know what realm
trillium.example.com is in, so it asked the \code{ATHENA.MIT.EDU} KDC for
a referral.  The next time you connect to \code{trillium.example.com},
the odd-looking entry will be used to avoid needing to ask for a
referral again.

You can use the \textbf{-f} option to view the flags that apply to your
tickets.  The flags are:

\begin{tabulary}{\linewidth}{|L|L|}
\hline

F
 & 
Forwardable
\\
\hline
f
 & 
forwarded
\\
\hline
P
 & 
Proxiable
\\
\hline
p
 & 
proxy
\\
\hline
D
 & 
postDateable
\\
\hline
d
 & 
postdated
\\
\hline
R
 & 
Renewable
\\
\hline
I
 & 
Initial
\\
\hline
i
 & 
invalid
\\
\hline
H
 & 
Hardware authenticated
\\
\hline
A
 & 
preAuthenticated
\\
\hline
T
 & 
Transit policy checked
\\
\hline
O
 & 
Okay as delegate
\\
\hline
a
 & 
anonymous
\\
\hline\end{tabulary}


Here is a sample listing.  In this example, the user \emph{jennifer}
obtained her initial tickets (\textbf{I}), which are forwardable (\textbf{F})
and postdated (\textbf{d}) but not yet validated (\textbf{i}):

\begin{Verbatim}[commandchars=\\\{\}]
shell\PYGZpc{} klist \PYGZhy{}f
Ticket cache: /tmp/krb5cc\PYGZus{}320
Default principal: jennifer@ATHENA.MIT.EDU

Valid starting      Expires             Service principal
31/07/05 19:06:25  31/07/05 19:16:25  krbtgt/ATHENA.MIT.EDU@ATHENA.MIT.EDU
        Flags: FdiI
shell\PYGZpc{}
\end{Verbatim}

In the following example, the user \emph{david}`s tickets were forwarded
(\textbf{f}) to this host from another host.  The tickets are reforwardable
(\textbf{F}):

\begin{Verbatim}[commandchars=\\\{\}]
shell\PYGZpc{} klist \PYGZhy{}f
Ticket cache: /tmp/krb5cc\PYGZus{}p11795
Default principal: david@EXAMPLE.COM

Valid starting     Expires            Service principal
07/31/05 11:52:29  07/31/05 21:11:23  krbtgt/EXAMPLE.COM@EXAMPLE.COM
        Flags: Ff
07/31/05 12:03:48  07/31/05 21:11:23  host/trillium.example.com@EXAMPLE.COM
        Flags: Ff
shell\PYGZpc{}
\end{Verbatim}


\section{Destroying tickets with kdestroy}
\label{user/tkt_mgmt:destroying-tickets-with-kdestroy}
Your Kerberos tickets are proof that you are indeed yourself, and
tickets could be stolen if someone gains access to a computer where
they are stored.  If this happens, the person who has them can
masquerade as you until they expire.  For this reason, you should
destroy your Kerberos tickets when you are away from your computer.

Destroying your tickets is easy.  Simply type kdestroy:

\begin{Verbatim}[commandchars=\\\{\}]
shell\PYGZpc{} kdestroy
shell\PYGZpc{}
\end{Verbatim}

If {\hyperref[user/user_commands/kdestroy:kdestroy-1]{\emph{kdestroy}}} fails to destroy your tickets, it will beep and
give an error message.  For example, if kdestroy can't find any
tickets to destroy, it will give the following message:

\begin{Verbatim}[commandchars=\\\{\}]
shell\PYGZpc{} kdestroy
kdestroy: No credentials cache file found while destroying cache
shell\PYGZpc{}
\end{Verbatim}


\chapter{User config files}
\label{user/user_config/index::doc}\label{user/user_config/index:user-config-files}
The following files in your home directory can be used to control the
behavior of Kerberos as it applies to your account (unless they have
been disabled by your host's configuration):


\section{.k5login}
\label{user/user_config/k5login:k5login-5}\label{user/user_config/k5login:k5login}\label{user/user_config/k5login::doc}

\subsection{DESCRIPTION}
\label{user/user_config/k5login:description}
The .k5login file, which resides in a user's home directory, contains
a list of the Kerberos principals.  Anyone with valid tickets for a
principal in the file is allowed host access with the UID of the user
in whose home directory the file resides.  One common use is to place
a .k5login file in root's home directory, thereby granting system
administrators remote root access to the host via Kerberos.


\subsection{EXAMPLES}
\label{user/user_config/k5login:examples}
Suppose the user \code{alice} had a .k5login file in her home directory
containing just the following line:

\begin{Verbatim}[commandchars=\\\{\}]
bob@FOOBAR.ORG
\end{Verbatim}

This would allow \code{bob} to use Kerberos network applications, such as
ssh(1), to access \code{alice}`s account, using \code{bob}`s Kerberos
tickets.  In a default configuration (with \textbf{k5login\_authoritative} set
to true in \emph{krb5.conf(5)}), this .k5login file would not let
\code{alice} use those network applications to access her account, since
she is not listed!  With no .k5login file, or with \textbf{k5login\_authoritative}
set to false, a default rule would permit the principal \code{alice} in the
machine's default realm to access the \code{alice} account.

Let us further suppose that \code{alice} is a system administrator.
Alice and the other system administrators would have their principals
in root's .k5login file on each host:

\begin{Verbatim}[commandchars=\\\{\}]
alice@BLEEP.COM

joeadmin/root@BLEEP.COM
\end{Verbatim}

This would allow either system administrator to log in to these hosts
using their Kerberos tickets instead of having to type the root
password.  Note that because \code{bob} retains the Kerberos tickets for
his own principal, \code{bob@FOOBAR.ORG}, he would not have any of the
privileges that require \code{alice}`s tickets, such as root access to
any of the site's hosts, or the ability to change \code{alice}`s
password.


\subsection{SEE ALSO}
\label{user/user_config/k5login:see-also}
kerberos(1)


\section{.k5identity}
\label{user/user_config/k5identity:k5identity-5}\label{user/user_config/k5identity:k5identity}\label{user/user_config/k5identity::doc}

\subsection{DESCRIPTION}
\label{user/user_config/k5identity:description}
The .k5identity file, which resides in a user's home directory,
contains a list of rules for selecting a client principals based on
the server being accessed.  These rules are used to choose a
credential cache within the cache collection when possible.

Blank lines and lines beginning with \code{\#} are ignored.  Each line has
the form:
\begin{quote}

\emph{principal} \emph{field}=\emph{value} ...
\end{quote}

If the server principal meets all of the field constraints, then
principal is chosen as the client principal.  The following fields are
recognized:
\begin{description}
\item[{\textbf{realm}}] \leavevmode
If the realm of the server principal is known, it is matched
against \emph{value}, which may be a pattern using shell wildcards.
For host-based server principals, the realm will generally only be
known if there is a \emph{domain\_realm} section in
\emph{krb5.conf(5)} with a mapping for the hostname.

\item[{\textbf{service}}] \leavevmode
If the server principal is a host-based principal, its service
component is matched against \emph{value}, which may be a pattern using
shell wildcards.

\item[{\textbf{host}}] \leavevmode
If the server principal is a host-based principal, its hostname
component is converted to lower case and matched against \emph{value},
which may be a pattern using shell wildcards.

If the server principal matches the constraints of multiple lines
in the .k5identity file, the principal from the first matching
line is used.  If no line matches, credentials will be selected
some other way, such as the realm heuristic or the current primary
cache.

\end{description}


\subsection{EXAMPLE}
\label{user/user_config/k5identity:example}
The following example .k5identity file selects the client principal
\code{alice@KRBTEST.COM} if the server principal is within that realm,
the principal \code{alice/root@EXAMPLE.COM} if the server host is within
a servers subdomain, and the principal \code{alice/mail@EXAMPLE.COM} when
accessing the IMAP service on \code{mail.example.com}:

\begin{Verbatim}[commandchars=\\\{\}]
alice@KRBTEST.COM       realm=KRBTEST.COM
alice/root@EXAMPLE.COM  host=*.servers.example.com
alice/mail@EXAMPLE.COM  host=mail.example.com service=imap
\end{Verbatim}


\subsection{SEE ALSO}
\label{user/user_config/k5identity:see-also}
kerberos(1), \emph{krb5.conf(5)}


\chapter{User commands}
\label{user/user_commands/index::doc}\label{user/user_commands/index:user-commands}\label{user/user_commands/index:id1}

\section{kdestroy}
\label{user/user_commands/kdestroy:kdestroy}\label{user/user_commands/kdestroy::doc}\label{user/user_commands/kdestroy:kdestroy-1}

\subsection{SYNOPSIS}
\label{user/user_commands/kdestroy:synopsis}
\textbf{kdestroy}
{[}\textbf{-A}{]}
{[}\textbf{-q}{]}
{[}\textbf{-c} \emph{cache\_name}{]}


\subsection{DESCRIPTION}
\label{user/user_commands/kdestroy:description}
The kdestroy utility destroys the user's active Kerberos authorization
tickets by overwriting and deleting the credentials cache that
contains them.  If the credentials cache is not specified, the default
credentials cache is destroyed.


\subsection{OPTIONS}
\label{user/user_commands/kdestroy:options}\begin{description}
\item[{\textbf{-A}}] \leavevmode
Destroys all caches in the collection, if a cache collection is
available.

\item[{\textbf{-q}}] \leavevmode
Run quietly.  Normally kdestroy beeps if it fails to destroy the
user's tickets.  The \textbf{-q} flag suppresses this behavior.

\item[{\textbf{-c} \emph{cache\_name}}] \leavevmode
Use \emph{cache\_name} as the credentials (ticket) cache name and
location; if this option is not used, the default cache name and
location are used.

The default credentials cache may vary between systems.  If the
\textbf{KRB5CCNAME} environment variable is set, its value is used to
name the default ticket cache.

\end{description}


\subsection{NOTE}
\label{user/user_commands/kdestroy:note}
Most installations recommend that you place the kdestroy command in
your .logout file, so that your tickets are destroyed automatically
when you log out.


\subsection{ENVIRONMENT}
\label{user/user_commands/kdestroy:environment}
kdestroy uses the following environment variable:
\begin{description}
\item[{\textbf{KRB5CCNAME}}] \leavevmode
Location of the default Kerberos 5 credentials (ticket) cache, in
the form \emph{type}:\emph{residual}.  If no \emph{type} prefix is present, the
\textbf{FILE} type is assumed.  The type of the default cache may
determine the availability of a cache collection; for instance, a
default cache of type \textbf{DIR} causes caches within the directory
to be present in the collection.

\end{description}


\subsection{FILES}
\label{user/user_commands/kdestroy:files}\begin{description}
\item[{\emph{DEFCCNAME}}] \leavevmode
Default location of Kerberos 5 credentials cache

\end{description}


\subsection{SEE ALSO}
\label{user/user_commands/kdestroy:see-also}
{\hyperref[user/user_commands/kinit:kinit-1]{\emph{kinit}}}, {\hyperref[user/user_commands/klist:klist-1]{\emph{klist}}}


\section{kinit}
\label{user/user_commands/kinit:kinit-1}\label{user/user_commands/kinit:kinit}\label{user/user_commands/kinit::doc}

\subsection{SYNOPSIS}
\label{user/user_commands/kinit:synopsis}
\textbf{kinit}
{[}\textbf{-V}{]}
{[}\textbf{-l} \emph{lifetime}{]}
{[}\textbf{-s} \emph{start\_time}{]}
{[}\textbf{-r} \emph{renewable\_life}{]}
{[}\textbf{-p} \textbar{} -\textbf{P}{]}
{[}\textbf{-f} \textbar{} -\textbf{F}{]}
{[}\textbf{-a}{]}
{[}\textbf{-A}{]}
{[}\textbf{-C}{]}
{[}\textbf{-E}{]}
{[}\textbf{-v}{]}
{[}\textbf{-R}{]}
{[}\textbf{-k} {[}-\textbf{t} \emph{keytab\_file}{]}{]}
{[}\textbf{-c} \emph{cache\_name}{]}
{[}\textbf{-n}{]}
{[}\textbf{-S} \emph{service\_name}{]}
{[}\textbf{-I} \emph{input\_ccache}{]}
{[}\textbf{-T} \emph{armor\_ccache}{]}
{[}\textbf{-X} \emph{attribute}{[}=\emph{value}{]}{]}
{[}\emph{principal}{]}


\subsection{DESCRIPTION}
\label{user/user_commands/kinit:description}
kinit obtains and caches an initial ticket-granting ticket for
\emph{principal}.  If \emph{principal} is absent, kinit chooses an appropriate
principal name based on existing credential cache contents or the
local username of the user invoking kinit.  Some options modify the
choice of principal name.


\subsection{OPTIONS}
\label{user/user_commands/kinit:options}\begin{description}
\item[{\textbf{-V}}] \leavevmode
display verbose output.

\item[{\textbf{-l} \emph{lifetime}}] \leavevmode
(\emph{duration} string.)  Requests a ticket with the lifetime
\emph{lifetime}.

For example, \code{kinit -l 5:30} or \code{kinit -l 5h30m}.

If the \textbf{-l} option is not specified, the default ticket lifetime
(configured by each site) is used.  Specifying a ticket lifetime
longer than the maximum ticket lifetime (configured by each site)
will not override the configured maximum ticket lifetime.

\item[{\textbf{-s} \emph{start\_time}}] \leavevmode
(\emph{duration} string.)  Requests a postdated ticket.  Postdated
tickets are issued with the \textbf{invalid} flag set, and need to be
resubmitted to the KDC for validation before use.

\emph{start\_time} specifies the duration of the delay before the ticket
can become valid.

\item[{\textbf{-r} \emph{renewable\_life}}] \leavevmode
(\emph{duration} string.)  Requests renewable tickets, with a total
lifetime of \emph{renewable\_life}.

\item[{\textbf{-f}}] \leavevmode
requests forwardable tickets.

\item[{\textbf{-F}}] \leavevmode
requests non-forwardable tickets.

\item[{\textbf{-p}}] \leavevmode
requests proxiable tickets.

\item[{\textbf{-P}}] \leavevmode
requests non-proxiable tickets.

\item[{\textbf{-a}}] \leavevmode
requests tickets restricted to the host's local address{[}es{]}.

\item[{\textbf{-A}}] \leavevmode
requests tickets not restricted by address.

\item[{\textbf{-C}}] \leavevmode
requests canonicalization of the principal name, and allows the
KDC to reply with a different client principal from the one
requested.

\item[{\textbf{-E}}] \leavevmode
treats the principal name as an enterprise name (implies the
\textbf{-C} option).

\item[{\textbf{-v}}] \leavevmode
requests that the ticket-granting ticket in the cache (with the
\textbf{invalid} flag set) be passed to the KDC for validation.  If the
ticket is within its requested time range, the cache is replaced
with the validated ticket.

\item[{\textbf{-R}}] \leavevmode
requests renewal of the ticket-granting ticket.  Note that an
expired ticket cannot be renewed, even if the ticket is still
within its renewable life.

Note that renewable tickets that have expired as reported by
{\hyperref[user/user_commands/klist:klist-1]{\emph{klist}}} may sometimes be renewed using this option,
because the KDC applies a grace period to account for client-KDC
clock skew.  See \emph{krb5.conf(5)} \textbf{clockskew} setting.

\item[{\textbf{-k} {[}\textbf{-i} \textbar{} \textbf{-t} \emph{keytab\_file}{]}}] \leavevmode
requests a ticket, obtained from a key in the local host's keytab.
The location of the keytab may be specified with the \textbf{-t}
\emph{keytab\_file} option, or with the \textbf{-i} option to specify the use
of the default client keytab; otherwise the default keytab will be
used.  By default, a host ticket for the local host is requested,
but any principal may be specified.  On a KDC, the special keytab
location \code{KDB:} can be used to indicate that kinit should open
the KDC database and look up the key directly.  This permits an
administrator to obtain tickets as any principal that supports
authentication based on the key.

\item[{\textbf{-n}}] \leavevmode
Requests anonymous processing.  Two types of anonymous principals
are supported.

For fully anonymous Kerberos, configure pkinit on the KDC and
configure \textbf{pkinit\_anchors} in the client's \emph{krb5.conf(5)}.
Then use the \textbf{-n} option with a principal of the form \code{@REALM}
(an empty principal name followed by the at-sign and a realm
name).  If permitted by the KDC, an anonymous ticket will be
returned.

A second form of anonymous tickets is supported; these
realm-exposed tickets hide the identity of the client but not the
client's realm.  For this mode, use \code{kinit -n} with a normal
principal name.  If supported by the KDC, the principal (but not
realm) will be replaced by the anonymous principal.

As of release 1.8, the MIT Kerberos KDC only supports fully
anonymous operation.

\end{description}

\textbf{-I} \emph{input\_ccache}
\begin{quote}

Specifies the name of a credentials cache that already contains a
ticket.  When obtaining that ticket, if information about how that
ticket was obtained was also stored to the cache, that information
will be used to affect how new credentials are obtained, including
preselecting the same methods of authenticating to the KDC.
\end{quote}
\begin{description}
\item[{\textbf{-T} \emph{armor\_ccache}}] \leavevmode
Specifies the name of a credentials cache that already contains a
ticket.  If supported by the KDC, this cache will be used to armor
the request, preventing offline dictionary attacks and allowing
the use of additional preauthentication mechanisms.  Armoring also
makes sure that the response from the KDC is not modified in
transit.

\item[{\textbf{-c} \emph{cache\_name}}] \leavevmode
use \emph{cache\_name} as the Kerberos 5 credentials (ticket) cache
location.  If this option is not used, the default cache location
is used.

The default cache location may vary between systems.  If the
\textbf{KRB5CCNAME} environment variable is set, its value is used to
locate the default cache.  If a principal name is specified and
the type of the default cache supports a collection (such as the
DIR type), an existing cache containing credentials for the
principal is selected or a new one is created and becomes the new
primary cache.  Otherwise, any existing contents of the default
cache are destroyed by kinit.

\item[{\textbf{-S} \emph{service\_name}}] \leavevmode
specify an alternate service name to use when getting initial
tickets.

\item[{\textbf{-X} \emph{attribute}{[}=\emph{value}{]}}] \leavevmode
specify a pre-authentication \emph{attribute} and \emph{value} to be
interpreted by pre-authentication modules.  The acceptable
attribute and value values vary from module to module.  This
option may be specified multiple times to specify multiple
attributes.  If no value is specified, it is assumed to be ``yes''.

The following attributes are recognized by the PKINIT
pre-authentication mechanism:
\begin{description}
\item[{\textbf{X509\_user\_identity}=\emph{value}}] \leavevmode
specify where to find user's X509 identity information

\item[{\textbf{X509\_anchors}=\emph{value}}] \leavevmode
specify where to find trusted X509 anchor information

\item[{\textbf{flag\_RSA\_PROTOCOL}{[}\textbf{=yes}{]}}] \leavevmode
specify use of RSA, rather than the default Diffie-Hellman
protocol

\end{description}

\end{description}


\subsection{ENVIRONMENT}
\label{user/user_commands/kinit:environment}
kinit uses the following environment variables:
\begin{description}
\item[{\textbf{KRB5CCNAME}}] \leavevmode
Location of the default Kerberos 5 credentials cache, in the form
\emph{type}:\emph{residual}.  If no \emph{type} prefix is present, the \textbf{FILE}
type is assumed.  The type of the default cache may determine the
availability of a cache collection; for instance, a default cache
of type \textbf{DIR} causes caches within the directory to be present
in the collection.

\end{description}


\subsection{FILES}
\label{user/user_commands/kinit:files}\begin{description}
\item[{\emph{DEFCCNAME}}] \leavevmode
default location of Kerberos 5 credentials cache

\item[{\emph{DEFKTNAME}}] \leavevmode
default location for the local host's keytab.

\end{description}


\subsection{SEE ALSO}
\label{user/user_commands/kinit:see-also}
{\hyperref[user/user_commands/klist:klist-1]{\emph{klist}}}, {\hyperref[user/user_commands/kdestroy:kdestroy-1]{\emph{kdestroy}}}, kerberos(1)


\section{klist}
\label{user/user_commands/klist:klist}\label{user/user_commands/klist::doc}\label{user/user_commands/klist:klist-1}

\subsection{SYNOPSIS}
\label{user/user_commands/klist:synopsis}
\textbf{klist}
{[}\textbf{-e}{]}
{[}{[}\textbf{-c}{]} {[}\textbf{-l}{]} {[}\textbf{-A}{]} {[}\textbf{-f}{]} {[}\textbf{-s}{]} {[}\textbf{-a} {[}\textbf{-n}{]}{]}{]}
{[}\textbf{-C}{]}
{[}\textbf{-k} {[}\textbf{-t}{]} {[}\textbf{-K}{]}{]}
{[}\textbf{-V}{]}
{[}\emph{cache\_name}\textbar{}\emph{keytab\_name}{]}


\subsection{DESCRIPTION}
\label{user/user_commands/klist:description}
klist lists the Kerberos principal and Kerberos tickets held in a
credentials cache, or the keys held in a keytab file.


\subsection{OPTIONS}
\label{user/user_commands/klist:options}\begin{description}
\item[{\textbf{-e}}] \leavevmode
Displays the encryption types of the session key and the ticket
for each credential in the credential cache, or each key in the
keytab file.

\item[{\textbf{-l}}] \leavevmode
If a cache collection is available, displays a table summarizing
the caches present in the collection.

\item[{\textbf{-A}}] \leavevmode
If a cache collection is available, displays the contents of all
of the caches in the collection.

\item[{\textbf{-c}}] \leavevmode
List tickets held in a credentials cache. This is the default if
neither \textbf{-c} nor \textbf{-k} is specified.

\item[{\textbf{-f}}] \leavevmode
Shows the flags present in the credentials, using the following
abbreviations:

\begin{Verbatim}[commandchars=\\\{\}]
F    Forwardable
f    forwarded
P    Proxiable
p    proxy
D    postDateable
d    postdated
R    Renewable
I    Initial
i    invalid
H    Hardware authenticated
A    preAuthenticated
T    Transit policy checked
O    Okay as delegate
a    anonymous
\end{Verbatim}

\item[{\textbf{-s}}] \leavevmode
Causes klist to run silently (produce no output).  klist will exit
with status 1 if the credentials cache cannot be read or is
expired, and with status 0 otherwise.

\item[{\textbf{-a}}] \leavevmode
Display list of addresses in credentials.

\item[{\textbf{-n}}] \leavevmode
Show numeric addresses instead of reverse-resolving addresses.

\item[{\textbf{-C}}] \leavevmode
List configuration data that has been stored in the credentials
cache when klist encounters it.  By default, configuration data
is not listed.

\item[{\textbf{-k}}] \leavevmode
List keys held in a keytab file.

\item[{\textbf{-i}}] \leavevmode
In combination with \textbf{-k}, defaults to using the default client
keytab instead of the default acceptor keytab, if no name is
given.

\item[{\textbf{-t}}] \leavevmode
Display the time entry timestamps for each keytab entry in the
keytab file.

\item[{\textbf{-K}}] \leavevmode
Display the value of the encryption key in each keytab entry in
the keytab file.

\item[{\textbf{-V}}] \leavevmode
Display the Kerberos version number and exit.

\end{description}

If \emph{cache\_name} or \emph{keytab\_name} is not specified, klist will display
the credentials in the default credentials cache or keytab file as
appropriate.  If the \textbf{KRB5CCNAME} environment variable is set, its
value is used to locate the default ticket cache.


\subsection{ENVIRONMENT}
\label{user/user_commands/klist:environment}
klist uses the following environment variable:
\begin{description}
\item[{\textbf{KRB5CCNAME}}] \leavevmode
Location of the default Kerberos 5 credentials (ticket) cache, in
the form \emph{type}:\emph{residual}.  If no \emph{type} prefix is present, the
\textbf{FILE} type is assumed.  The type of the default cache may
determine the availability of a cache collection; for instance, a
default cache of type \textbf{DIR} causes caches within the directory
to be present in the collection.

\end{description}


\subsection{FILES}
\label{user/user_commands/klist:files}\begin{description}
\item[{\emph{DEFCCNAME}}] \leavevmode
Default location of Kerberos 5 credentials cache

\item[{\emph{DEFKTNAME}}] \leavevmode
Default location for the local host's keytab file.

\end{description}


\subsection{SEE ALSO}
\label{user/user_commands/klist:see-also}
{\hyperref[user/user_commands/kinit:kinit-1]{\emph{kinit}}}, {\hyperref[user/user_commands/kdestroy:kdestroy-1]{\emph{kdestroy}}}


\section{kpasswd}
\label{user/user_commands/kpasswd:kpasswd}\label{user/user_commands/kpasswd::doc}\label{user/user_commands/kpasswd:kpasswd-1}

\subsection{SYNOPSIS}
\label{user/user_commands/kpasswd:synopsis}
\textbf{kpasswd} {[}\emph{principal}{]}


\subsection{DESCRIPTION}
\label{user/user_commands/kpasswd:description}
The kpasswd command is used to change a Kerberos principal's password.
kpasswd first prompts for the current Kerberos password, then prompts
the user twice for the new password, and the password is changed.

If the principal is governed by a policy that specifies the length
and/or number of character classes required in the new password, the
new password must conform to the policy.  (The five character classes
are lower case, upper case, numbers, punctuation, and all other
characters.)


\subsection{OPTIONS}
\label{user/user_commands/kpasswd:options}\begin{description}
\item[{\emph{principal}}] \leavevmode
Change the password for the Kerberos principal principal.
Otherwise, kpasswd uses the principal name from an existing ccache
if there is one; if not, the principal is derived from the
identity of the user invoking the kpasswd command.

\end{description}


\subsection{SEE ALSO}
\label{user/user_commands/kpasswd:see-also}
\emph{kadmin(1)}, \emph{kadmind(8)}


\section{krb5-config}
\label{user/user_commands/krb5-config:krb5-config-1}\label{user/user_commands/krb5-config:krb5-config}\label{user/user_commands/krb5-config::doc}

\subsection{SYNOPSIS}
\label{user/user_commands/krb5-config:synopsis}
\textbf{krb5-config}
{[}\textbf{-}\textbf{-help} \textbar{} \textbf{-}\textbf{-all} \textbar{} \textbf{-}\textbf{-version} \textbar{} \textbf{-}\textbf{-vendor} \textbar{} \textbf{-}\textbf{-prefix} \textbar{} \textbf{-}\textbf{-exec-prefix} \textbar{} \textbf{-}\textbf{-defccname} \textbar{} \textbf{-}\textbf{-defktname} \textbar{} \textbf{-}\textbf{-defcktname} \textbar{} \textbf{-}\textbf{-cflags} \textbar{} \textbf{-}\textbf{-libs} {[}\emph{libraries}{]}{]}


\subsection{DESCRIPTION}
\label{user/user_commands/krb5-config:description}
krb5-config tells the application programmer what flags to use to compile
and link programs against the installed Kerberos libraries.


\subsection{OPTIONS}
\label{user/user_commands/krb5-config:options}\begin{description}
\item[{\textbf{-}\textbf{-help}}] \leavevmode
prints a usage message.  This is the default behavior when no options
are specified.

\item[{\textbf{-}\textbf{-all}}] \leavevmode
prints the version, vendor, prefix, and exec-prefix.

\item[{\textbf{-}\textbf{-version}}] \leavevmode
prints the version number of the Kerberos installation.

\item[{\textbf{-}\textbf{-vendor}}] \leavevmode
prints the name of the vendor of the Kerberos installation.

\item[{\textbf{-}\textbf{-prefix}}] \leavevmode
prints the prefix for which the Kerberos installation was built.

\item[{\textbf{-}\textbf{-exec-prefix}}] \leavevmode
prints the prefix for executables for which the Kerberos installation
was built.

\item[{\textbf{-}\textbf{-defccname}}] \leavevmode
prints the built-in default credentials cache location.

\item[{\textbf{-}\textbf{-defktname}}] \leavevmode
prints the built-in default keytab location.

\item[{\textbf{-}\textbf{-defcktname}}] \leavevmode
prints the built-in default client (initiator) keytab location.

\item[{\textbf{-}\textbf{-cflags}}] \leavevmode
prints the compilation flags used to build the Kerberos installation.

\item[{\textbf{-}\textbf{-libs} {[}\emph{library}{]}}] \leavevmode
prints the compiler options needed to link against \emph{library}.
Allowed values for \emph{library} are:

\begin{tabulary}{\linewidth}{|L|L|}
\hline

krb5
 & 
Kerberos 5 applications (default)
\\
\hline
gssapi
 & 
GSSAPI applications with Kerberos 5 bindings
\\
\hline
kadm-client
 & 
Kadmin client
\\
\hline
kadm-server
 & 
Kadmin server
\\
\hline
kdb
 & 
Applications that access the Kerberos database
\\
\hline\end{tabulary}


\end{description}


\subsection{EXAMPLES}
\label{user/user_commands/krb5-config:examples}
krb5-config is particularly useful for compiling against a Kerberos
installation that was installed in a non-standard location.  For example,
a Kerberos installation that is installed in \code{/opt/krb5/} but uses
libraries in \code{/usr/local/lib/} for text localization would produce
the following output:

\begin{Verbatim}[commandchars=\\\{\}]
shell\PYGZpc{} krb5\PYGZhy{}config \PYGZhy{}\PYGZhy{}libs krb5
\PYGZhy{}L/opt/krb5/lib \PYGZhy{}Wl,\PYGZhy{}rpath \PYGZhy{}Wl,/opt/krb5/lib \PYGZhy{}L/usr/local/lib \PYGZhy{}lkrb5 \PYGZhy{}lk5crypto \PYGZhy{}lcom\PYGZus{}err
\end{Verbatim}


\subsection{SEE ALSO}
\label{user/user_commands/krb5-config:see-also}
kerberos(1), cc(1)


\section{ksu}
\label{user/user_commands/ksu:ksu-1}\label{user/user_commands/ksu:ksu}\label{user/user_commands/ksu::doc}

\subsection{SYNOPSIS}
\label{user/user_commands/ksu:synopsis}
\textbf{ksu}
{[} \emph{target\_user} {]}
{[} \textbf{-n} \emph{target\_principal\_name} {]}
{[} \textbf{-c} \emph{source\_cache\_name} {]}
{[} \textbf{-k} {]}
{[} \textbf{-r} time {]}
{[} \textbf{-pf} {]}
{[} \textbf{-l} \emph{lifetime} {]}
{[} \textbf{-z \textbar{} Z} {]}
{[} \textbf{-q} {]}
{[} \textbf{-e} \emph{command} {[} args ...  {]} {]} {[} \textbf{-a} {[} args ...  {]} {]}


\subsection{REQUIREMENTS}
\label{user/user_commands/ksu:requirements}
Must have Kerberos version 5 installed to compile ksu.  Must have a
Kerberos version 5 server running to use ksu.


\subsection{DESCRIPTION}
\label{user/user_commands/ksu:description}
ksu is a Kerberized version of the su program that has two missions:
one is to securely change the real and effective user ID to that of
the target user, and the other is to create a new security context.

\begin{notice}{note}{Note:}
For the sake of clarity, all references to and attributes of
the user invoking the program will start with ``source''
(e.g., ``source user'', ``source cache'', etc.).

Likewise, all references to and attributes of the target
account will start with ``target''.
\end{notice}


\subsection{AUTHENTICATION}
\label{user/user_commands/ksu:authentication}
To fulfill the first mission, ksu operates in two phases:
authentication and authorization.  Resolving the target principal name
is the first step in authentication.  The user can either specify his
principal name with the \textbf{-n} option (e.g., \code{-n jqpublic@USC.EDU})
or a default principal name will be assigned using a heuristic
described in the OPTIONS section (see \textbf{-n} option).  The target user
name must be the first argument to ksu; if not specified root is the
default.  If \code{.} is specified then the target user will be the
source user (e.g., \code{ksu .}).  If the source user is root or the
target user is the source user, no authentication or authorization
takes place.  Otherwise, ksu looks for an appropriate Kerberos ticket
in the source cache.

The ticket can either be for the end-server or a ticket granting
ticket (TGT) for the target principal's realm.  If the ticket for the
end-server is already in the cache, it's decrypted and verified.  If
it's not in the cache but the TGT is, the TGT is used to obtain the
ticket for the end-server.  The end-server ticket is then verified.
If neither ticket is in the cache, but ksu is compiled with the
\textbf{GET\_TGT\_VIA\_PASSWD} define, the user will be prompted for a
Kerberos password which will then be used to get a TGT.  If the user
is logged in remotely and does not have a secure channel, the password
may be exposed.  If neither ticket is in the cache and
\textbf{GET\_TGT\_VIA\_PASSWD} is not defined, authentication fails.


\subsection{AUTHORIZATION}
\label{user/user_commands/ksu:authorization}
This section describes authorization of the source user when ksu is
invoked without the \textbf{-e} option.  For a description of the \textbf{-e}
option, see the OPTIONS section.

Upon successful authentication, ksu checks whether the target
principal is authorized to access the target account.  In the target
user's home directory, ksu attempts to access two authorization files:
{\hyperref[user/user_config/k5login:k5login-5]{\emph{.k5login}}} and .k5users.  In the .k5login file each line
contains the name of a principal that is authorized to access the
account.

For example:

\begin{Verbatim}[commandchars=\\\{\}]
jqpublic@USC.EDU
jqpublic/secure@USC.EDU
jqpublic/admin@USC.EDU
\end{Verbatim}

The format of .k5users is the same, except the principal name may be
followed by a list of commands that the principal is authorized to
execute (see the \textbf{-e} option in the OPTIONS section for details).

Thus if the target principal name is found in the .k5login file the
source user is authorized to access the target account.  Otherwise ksu
looks in the .k5users file.  If the target principal name is found
without any trailing commands or followed only by \code{*} then the
source user is authorized.  If either .k5login or .k5users exist but
an appropriate entry for the target principal does not exist then
access is denied.  If neither file exists then the principal will be
granted access to the account according to the aname-\textgreater{}lname mapping
rules.  Otherwise, authorization fails.


\subsection{EXECUTION OF THE TARGET SHELL}
\label{user/user_commands/ksu:execution-of-the-target-shell}
Upon successful authentication and authorization, ksu proceeds in a
similar fashion to su.  The environment is unmodified with the
exception of USER, HOME and SHELL variables.  If the target user is
not root, USER gets set to the target user name.  Otherwise USER
remains unchanged.  Both HOME and SHELL are set to the target login's
default values.  In addition, the environment variable \textbf{KRB5CCNAME}
gets set to the name of the target cache.  The real and effective user
ID are changed to that of the target user.  The target user's shell is
then invoked (the shell name is specified in the password file).  Upon
termination of the shell, ksu deletes the target cache (unless ksu is
invoked with the \textbf{-k} option).  This is implemented by first doing a
fork and then an exec, instead of just exec, as done by su.


\subsection{CREATING A NEW SECURITY CONTEXT}
\label{user/user_commands/ksu:creating-a-new-security-context}
ksu can be used to create a new security context for the target
program (either the target shell, or command specified via the \textbf{-e}
option).  The target program inherits a set of credentials from the
source user.  By default, this set includes all of the credentials in
the source cache plus any additional credentials obtained during
authentication.  The source user is able to limit the credentials in
this set by using \textbf{-z} or \textbf{-Z} option.  \textbf{-z} restricts the copy
of tickets from the source cache to the target cache to only the
tickets where client == the target principal name.  The \textbf{-Z} option
provides the target user with a fresh target cache (no creds in the
cache).  Note that for security reasons, when the source user is root
and target user is non-root, \textbf{-z} option is the default mode of
operation.

While no authentication takes place if the source user is root or is
the same as the target user, additional tickets can still be obtained
for the target cache.  If \textbf{-n} is specified and no credentials can
be copied to the target cache, the source user is prompted for a
Kerberos password (unless \textbf{-Z} specified or \textbf{GET\_TGT\_VIA\_PASSWD}
is undefined).  If successful, a TGT is obtained from the Kerberos
server and stored in the target cache.  Otherwise, if a password is
not provided (user hit return) ksu continues in a normal mode of
operation (the target cache will not contain the desired TGT).  If the
wrong password is typed in, ksu fails.

\begin{notice}{note}{Note:}
During authentication, only the tickets that could be
obtained without providing a password are cached in in the
source cache.
\end{notice}


\subsection{OPTIONS}
\label{user/user_commands/ksu:options}\begin{description}
\item[{\textbf{-n} \emph{target\_principal\_name}}] \leavevmode
Specify a Kerberos target principal name.  Used in authentication
and authorization phases of ksu.

If ksu is invoked without \textbf{-n}, a default principal name is
assigned via the following heuristic:
\begin{itemize}
\item {} 
Case 1: source user is non-root.

If the target user is the source user the default principal name
is set to the default principal of the source cache.  If the
cache does not exist then the default principal name is set to
\code{target\_user@local\_realm}.  If the source and target users are
different and neither \code{\textasciitilde{}target\_user/.k5users} nor
\code{\textasciitilde{}target\_user/.k5login} exist then the default principal name
is \code{target\_user\_login\_name@local\_realm}.  Otherwise, starting
with the first principal listed below, ksu checks if the
principal is authorized to access the target account and whether
there is a legitimate ticket for that principal in the source
cache.  If both conditions are met that principal becomes the
default target principal, otherwise go to the next principal.
\begin{enumerate}
\item {} 
default principal of the source cache

\item {} 
target\_user@local\_realm

\item {} 
source\_user@local\_realm

\end{enumerate}

If a-c fails try any principal for which there is a ticket in
the source cache and that is authorized to access the target
account.  If that fails select the first principal that is
authorized to access the target account from the above list.  If
none are authorized and ksu is configured with
\textbf{PRINC\_LOOK\_AHEAD} turned on, select the default principal as
follows:

For each candidate in the above list, select an authorized
principal that has the same realm name and first part of the
principal name equal to the prefix of the candidate.  For
example if candidate a) is \code{jqpublic@ISI.EDU} and
\code{jqpublic/secure@ISI.EDU} is authorized to access the target
account then the default principal is set to
\code{jqpublic/secure@ISI.EDU}.

\item {} 
Case 2: source user is root.

If the target user is non-root then the default principal name
is \code{target\_user@local\_realm}.  Else, if the source cache
exists the default principal name is set to the default
principal of the source cache.  If the source cache does not
exist, default principal name is set to \code{root\textbackslash{}@local\_realm}.

\end{itemize}

\end{description}

\textbf{-c} \emph{source\_cache\_name}
\begin{quote}

Specify source cache name (e.g., \code{-c FILE:/tmp/my\_cache}).  If
\textbf{-c} option is not used then the name is obtained from
\textbf{KRB5CCNAME} environment variable.  If \textbf{KRB5CCNAME} is not
defined the source cache name is set to \code{krb5cc\_\textless{}source uid\textgreater{}}.
The target cache name is automatically set to \code{krb5cc\_\textless{}target
uid\textgreater{}.(gen\_sym())}, where gen\_sym generates a new number such that
the resulting cache does not already exist.  For example:

\begin{Verbatim}[commandchars=\\\{\}]
krb5cc\PYGZus{}1984.2
\end{Verbatim}
\end{quote}
\begin{description}
\item[{\textbf{-k}}] \leavevmode
Do not delete the target cache upon termination of the target
shell or a command (\textbf{-e} command).  Without \textbf{-k}, ksu deletes
the target cache.

\item[{\textbf{-z}}] \leavevmode
Restrict the copy of tickets from the source cache to the target
cache to only the tickets where client == the target principal
name.  Use the \textbf{-n} option if you want the tickets for other then
the default principal.  Note that the \textbf{-z} option is mutually
exclusive with the \textbf{-Z} option.

\item[{\textbf{-Z}}] \leavevmode
Don't copy any tickets from the source cache to the target cache.
Just create a fresh target cache, where the default principal name
of the cache is initialized to the target principal name.  Note
that the \textbf{-Z} option is mutually exclusive with the \textbf{-z}
option.

\item[{\textbf{-q}}] \leavevmode
Suppress the printing of status messages.

\end{description}

Ticket granting ticket options:
\begin{description}
\item[{\textbf{-l} \emph{lifetime} \textbf{-r} \emph{time} \textbf{-pf}}] \leavevmode
The ticket granting ticket options only apply to the case where
there are no appropriate tickets in the cache to authenticate the
source user.  In this case if ksu is configured to prompt users
for a Kerberos password (\textbf{GET\_TGT\_VIA\_PASSWD} is defined), the
ticket granting ticket options that are specified will be used
when getting a ticket granting ticket from the Kerberos server.

\item[{\textbf{-l} \emph{lifetime}}] \leavevmode
(\emph{duration} string.)  Specifies the lifetime to be requested
for the ticket; if this option is not specified, the default ticket
lifetime (12 hours) is used instead.

\item[{\textbf{-r} \emph{time}}] \leavevmode
(\emph{duration} string.)  Specifies that the \textbf{renewable} option
should be requested for the ticket, and specifies the desired
total lifetime of the ticket.

\item[{\textbf{-p}}] \leavevmode
specifies that the \textbf{proxiable} option should be requested for
the ticket.

\item[{\textbf{-f}}] \leavevmode
option specifies that the \textbf{forwardable} option should be
requested for the ticket.

\item[{\textbf{-e} \emph{command} {[}\emph{args} ...{]}}] \leavevmode
ksu proceeds exactly the same as if it was invoked without the
\textbf{-e} option, except instead of executing the target shell, ksu
executes the specified command. Example of usage:

\begin{Verbatim}[commandchars=\\\{\}]
ksu bob \PYGZhy{}e ls \PYGZhy{}lag
\end{Verbatim}

The authorization algorithm for \textbf{-e} is as follows:

If the source user is root or source user == target user, no
authorization takes place and the command is executed.  If source
user id != 0, and \code{\textasciitilde{}target\_user/.k5users} file does not exist,
authorization fails.  Otherwise, \code{\textasciitilde{}target\_user/.k5users} file
must have an appropriate entry for target principal to get
authorized.

The .k5users file format:

A single principal entry on each line that may be followed by a
list of commands that the principal is authorized to execute.  A
principal name followed by a \code{*} means that the user is
authorized to execute any command.  Thus, in the following
example:

\begin{Verbatim}[commandchars=\\\{\}]
jqpublic@USC.EDU ls mail /local/kerberos/klist
jqpublic/secure@USC.EDU *
jqpublic/admin@USC.EDU
\end{Verbatim}

\code{jqpublic@USC.EDU} is only authorized to execute \code{ls},
\code{mail} and \code{klist} commands.  \code{jqpublic/secure@USC.EDU} is
authorized to execute any command.  \code{jqpublic/admin@USC.EDU} is
not authorized to execute any command.  Note, that
\code{jqpublic/admin@USC.EDU} is authorized to execute the target
shell (regular ksu, without the \textbf{-e} option) but
\code{jqpublic@USC.EDU} is not.

The commands listed after the principal name must be either a full
path names or just the program name.  In the second case,
\textbf{CMD\_PATH} specifying the location of authorized programs must
be defined at the compilation time of ksu.  Which command gets
executed?

If the source user is root or the target user is the source user
or the user is authorized to execute any command (\code{*} entry)
then command can be either a full or a relative path leading to
the target program.  Otherwise, the user must specify either a
full path or just the program name.

\item[{\textbf{-a} \emph{args}}] \leavevmode
Specify arguments to be passed to the target shell.  Note that all
flags and parameters following -a will be passed to the shell,
thus all options intended for ksu must precede \textbf{-a}.

The \textbf{-a} option can be used to simulate the \textbf{-e} option if
used as follows:

\begin{Verbatim}[commandchars=\\\{\}]
\PYGZhy{}a \PYGZhy{}c [command [arguments]].
\end{Verbatim}

\textbf{-c} is interpreted by the c-shell to execute the command.

\end{description}


\subsection{INSTALLATION INSTRUCTIONS}
\label{user/user_commands/ksu:installation-instructions}
ksu can be compiled with the following four flags:
\begin{description}
\item[{\textbf{GET\_TGT\_VIA\_PASSWD}}] \leavevmode
In case no appropriate tickets are found in the source cache, the
user will be prompted for a Kerberos password.  The password is
then used to get a ticket granting ticket from the Kerberos
server.  The danger of configuring ksu with this macro is if the
source user is logged in remotely and does not have a secure
channel, the password may get exposed.

\item[{\textbf{PRINC\_LOOK\_AHEAD}}] \leavevmode
During the resolution of the default principal name,
\textbf{PRINC\_LOOK\_AHEAD} enables ksu to find principal names in
the .k5users file as described in the OPTIONS section
(see \textbf{-n} option).

\item[{\textbf{CMD\_PATH}}] \leavevmode
Specifies a list of directories containing programs that users are
authorized to execute (via .k5users file).

\item[{\textbf{HAVE\_GETUSERSHELL}}] \leavevmode
If the source user is non-root, ksu insists that the target user's
shell to be invoked is a ``legal shell''.  \emph{getusershell(3)} is
called to obtain the names of ``legal shells''.  Note that the
target user's shell is obtained from the passwd file.

\end{description}

Sample configuration:

\begin{Verbatim}[commandchars=\\\{\}]
KSU\PYGZus{}OPTS = \PYGZhy{}DGET\PYGZus{}TGT\PYGZus{}VIA\PYGZus{}PASSWD \PYGZhy{}DPRINC\PYGZus{}LOOK\PYGZus{}AHEAD \PYGZhy{}DCMD\PYGZus{}PATH=\PYGZsq{}\PYGZdq{}/bin /usr/ucb /local/bin\PYGZdq{}
\end{Verbatim}

ksu should be owned by root and have the set user id bit turned on.

ksu attempts to get a ticket for the end server just as Kerberized
telnet and rlogin.  Thus, there must be an entry for the server in the
Kerberos database (e.g., \code{host/nii.isi.edu@ISI.EDU}).  The keytab
file must be in an appropriate location.


\subsection{SIDE EFFECTS}
\label{user/user_commands/ksu:side-effects}
ksu deletes all expired tickets from the source cache.


\subsection{AUTHOR OF KSU}
\label{user/user_commands/ksu:author-of-ksu}
GENNADY (ARI) MEDVINSKY


\section{kswitch}
\label{user/user_commands/kswitch:kswitch-1}\label{user/user_commands/kswitch:kswitch}\label{user/user_commands/kswitch::doc}

\subsection{SYNOPSIS}
\label{user/user_commands/kswitch:synopsis}
\textbf{kswitch}
\{\textbf{-c} \emph{cachename}\textbar{}\textbf{-p} \emph{principal}\}


\subsection{DESCRIPTION}
\label{user/user_commands/kswitch:description}
kswitch makes the specified credential cache the primary cache for the
collection, if a cache collection is available.


\subsection{OPTIONS}
\label{user/user_commands/kswitch:options}\begin{description}
\item[{\textbf{-c} \emph{cachename}}] \leavevmode
Directly specifies the credential cache to be made primary.

\item[{\textbf{-p} \emph{principal}}] \leavevmode
Causes the cache collection to be searched for a cache containing
credentials for \emph{principal}.  If one is found, that collection is
made primary.

\end{description}


\subsection{ENVIRONMENT}
\label{user/user_commands/kswitch:environment}
kswitch uses the following environment variables:
\begin{description}
\item[{\textbf{KRB5CCNAME}}] \leavevmode
Location of the default Kerberos 5 credentials (ticket) cache, in
the form \emph{type}:\emph{residual}.  If no \emph{type} prefix is present, the
\textbf{FILE} type is assumed.  The type of the default cache may
determine the availability of a cache collection; for instance, a
default cache of type \textbf{DIR} causes caches within the directory
to be present in the collection.

\end{description}


\subsection{FILES}
\label{user/user_commands/kswitch:files}\begin{description}
\item[{\emph{DEFCCNAME}}] \leavevmode
Default location of Kerberos 5 credentials cache

\end{description}


\subsection{SEE ALSO}
\label{user/user_commands/kswitch:see-also}
{\hyperref[user/user_commands/kinit:kinit-1]{\emph{kinit}}}, {\hyperref[user/user_commands/kdestroy:kdestroy-1]{\emph{kdestroy}}}, {\hyperref[user/user_commands/klist:klist-1]{\emph{klist}}}), kerberos(1)


\section{kvno}
\label{user/user_commands/kvno:kvno-1}\label{user/user_commands/kvno::doc}\label{user/user_commands/kvno:kvno}

\subsection{SYNOPSIS}
\label{user/user_commands/kvno:synopsis}
\textbf{kvno}
{[}\textbf{-c} \emph{ccache}{]}
{[}\textbf{-e} \emph{etype}{]}
{[}\textbf{-q}{]}
{[}\textbf{-h}{]}
{[}\textbf{-P}{]}
{[}\textbf{-S} \emph{sname}{]}
{[}\textbf{-U} \emph{for\_user}{]}
\emph{service1 service2} ...


\subsection{DESCRIPTION}
\label{user/user_commands/kvno:description}
kvno acquires a service ticket for the specified Kerberos principals
and prints out the key version numbers of each.


\subsection{OPTIONS}
\label{user/user_commands/kvno:options}\begin{description}
\item[{\textbf{-c} \emph{ccache}}] \leavevmode
Specifies the name of a credentials cache to use (if not the
default)

\item[{\textbf{-e} \emph{etype}}] \leavevmode
Specifies the enctype which will be requested for the session key
of all the services named on the command line.  This is useful in
certain backward compatibility situations.

\item[{\textbf{-q}}] \leavevmode
Suppress printing output when successful.  If a service ticket
cannot be obtained, an error message will still be printed and
kvno will exit with nonzero status.

\item[{\textbf{-h}}] \leavevmode
Prints a usage statement and exits.

\item[{\textbf{-P}}] \leavevmode
Specifies that the \emph{service1 service2} ...  arguments are to be
treated as services for which credentials should be acquired using
constrained delegation.  This option is only valid when used in
conjunction with protocol transition.

\item[{\textbf{-S} \emph{sname}}] \leavevmode
Specifies that the \emph{service1 service2} ... arguments are
interpreted as hostnames, and the service principals are to be
constructed from those hostnames and the service name \emph{sname}.
The service hostnames will be canonicalized according to the usual
rules for constructing service principals.

\item[{\textbf{-U} \emph{for\_user}}] \leavevmode
Specifies that protocol transition (S4U2Self) is to be used to
acquire a ticket on behalf of \emph{for\_user}.  If constrained
delegation is not requested, the service name must match the
credentials cache client principal.

\end{description}


\subsection{ENVIRONMENT}
\label{user/user_commands/kvno:environment}
kvno uses the following environment variable:
\begin{description}
\item[{\textbf{KRB5CCNAME}}] \leavevmode
Location of the credentials (ticket) cache.

\end{description}


\subsection{FILES}
\label{user/user_commands/kvno:files}\begin{description}
\item[{\emph{DEFCCNAME}}] \leavevmode
Default location of the credentials cache

\end{description}


\subsection{SEE ALSO}
\label{user/user_commands/kvno:see-also}
{\hyperref[user/user_commands/kinit:kinit-1]{\emph{kinit}}}, {\hyperref[user/user_commands/kdestroy:kdestroy-1]{\emph{kdestroy}}}


\section{sclient}
\label{user/user_commands/sclient:sclient}\label{user/user_commands/sclient::doc}\label{user/user_commands/sclient:sclient-1}

\subsection{SYNOPSIS}
\label{user/user_commands/sclient:synopsis}
\textbf{sclient} \emph{remotehost}


\subsection{DESCRIPTION}
\label{user/user_commands/sclient:description}
sclient is a sample application, primarily useful for testing
purposes.  It contacts a sample server \emph{sserver(8)} and
authenticates to it using Kerberos version 5 tickets, then displays
the server's response.


\subsection{SEE ALSO}
\label{user/user_commands/sclient:see-also}
{\hyperref[user/user_commands/kinit:kinit-1]{\emph{kinit}}}, \emph{sserver(8)}



\renewcommand{\indexname}{Index}
\printindex
\end{document}
